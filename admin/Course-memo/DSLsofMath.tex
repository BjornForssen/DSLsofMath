%!TEX TS-program = latex
\documentclass[10pt,a4paper]{article}
\usepackage[utf8]{inputenc}
\usepackage{geometry}
\geometry{includeheadfoot,margin=2cm}
\usepackage[english]{babel}
\usepackage{enumitem}
\usepackage{multicol}
\setlist[itemize]{leftmargin=*}
\setlength{\parindent}{0pt}
\setlength{\parskip}{6pt plus 2pt minus 1pt}
\pagestyle{plain}



\usepackage[breaklinks=true]{hyperref}
\hypersetup{colorlinks,%
citecolor=blue,%
filecolor=blue,%
linkcolor=blue,%
urlcolor=blue}


\setcounter{secnumdepth}{0}



\title{Domain-Specific Languages of Mathematics \\\\ \small Course-memo for
the 2018 instance of a 7.5hec BSc course at Chalmers and GU.}
\author{Patrik Jansson}

\begin{document}

\maketitle
\begin{multicols}{2}


\providecommand{\tightlist}{%
  \setlength{\itemsep}{0pt}\setlength{\parskip}{0pt}}

  \url{http://github.com/DSLsofMath/DSLsofMath/}

  Course codes: DAT326 / DIT982

  \subsection{Course team}\label{course-team}

  \begin{itemize}
  \tightlist
  \item
    Examiner\&main lecturer: Patrik Jansson (patrikj@)
  \item
    Guest lecturer: Cezar Ionescu (cezar@)
  \item
    Teaching assistant: Daniel Schoepe (schoepe@)
  \end{itemize}

  \subsection{Objectives}\label{objectives}

  The course presents classical mathematical topics from a computing
  science perspective: giving specifications of the concepts introduced,
  paying attention to syntax and types, and ultimately constructing DSLs
  of some mathematical areas mentioned below.

  Learning outcomes as in the
  \href{https://www.student.chalmers.se/sp/course?course_id=26170}{course
  syllabus}.

  \begin{itemize}
  \tightlist
  \item
    Knowledge and understanding

    \begin{itemize}
    \tightlist
    \item
      design and implement a DSL (Domain-Specific Language) for a new
      domain
    \item
      organize areas of mathematics in DSL terms
    \item
      explain main concepts of elementary real and complex analysis,
      algebra, and linear algebra
    \end{itemize}
  \item
    Skills and abilities

    \begin{itemize}
    \tightlist
    \item
      develop adequate notation for mathematical concepts
    \item
      perform calculational proofs
    \item
      use power series for solving differential equations
    \item
      use Laplace transforms for solving differential equations
    \end{itemize}
  \item
    Judgement and approach

    \begin{itemize}
    \tightlist
    \item
      discuss and compare different software implementations of
      mathematical concepts
    \end{itemize}
  \end{itemize}

  The course is elective for both computer science and mathematics
  students at both Chalmers and GU.

  \subsection{Course material}\label{course-material}

  Lecture notes are freely available online. These notes + references
  therein cover the course (but there is no printed course textbook).

  \subsection{Course setup}\label{course-setup}

  \begin{itemize}
  \tightlist
  \item
    Lectures (Tue 13-15 and Thu 13-15 in EB)

    \begin{itemize}
    \tightlist
    \item
      Introduction: Haskell, complex numbers, syntax, semantics,
      evaluation, approximation
    \item
      Basic concepts of analysis: sequences, limits, convergence, \ldots{}
    \item
      Types and mathematics: logic, quantifiers, proofs and programs,
      Curry--Howard, \ldots{}
    \item
      Type classes, derivatives, derivation,
    \item
      Domain-Specific Languages and algebraic structures, algebras,
      homomorphisms
    \item
      Polynomials, series, power series
    \item
      Power series and differential equations, exp, sin, log, Taylor
      series, \ldots{}
    \item
      Linear algebra
    \item
      Laplace transform
    \end{itemize}
  \item
    Exercise sessions (Tue 15-17 and Thu 15-17 in ES52)

    \begin{itemize}
    \tightlist
    \item
      Half time helping students solve problems in small groups
    \item
      Half time joint problem solving at the whiteboard
    \end{itemize}
  \end{itemize}

  \subsection{Changes from last year}\label{changes-from-last-year}

  The main changes for 2018 (based on the \href{eval/2017-04-28.md}{course
  eval meeting}) are

  \begin{itemize}
  \tightlist
  \item
    New course literature (complete lecture notes)
  \item
    Developed more exercises to solve (primarily easier exercises to start
    each week with)
  \item
    Weekly hand-ins to encourage students to spend more hours on the
    course
  \item
    More solving of exercises at the whiteboard
  \item
    Schedule changes (alternating L, E, L, E instead of L, L, E, E)
  \end{itemize}

  \subsection{Examination}\label{examination}

  There are two compulsory course elements:

  \begin{itemize}
  \tightlist
  \item
    Assignments (written + oral examination in groups of three students)

    \begin{itemize}
    \tightlist
    \item
      two compulsory hand-in assignments (2018-01-30, 2018-02-27)
    \item
      Grading: Pass or fail
    \end{itemize}
  \item
    Exam (individual written exam at the end of the course)

    \begin{itemize}
    \tightlist
    \item
      Grading: Chalmers: U, 3, 4, 5; GU: U, G, VG
    \item
      Date:
      \href{https://www.student.chalmers.se/sp/course?course_id=26170}{2018-03-13
      at 14.00}
    \item
      Aids: One textbook of your choice
    \end{itemize}
  \end{itemize}

  To pass the course you need to pass both course elements.

\end{multicols}
\end{document}
